\documentclass[11pt,a4paper,sans]{moderncv}

%% ModernCV themes
\moderncvstyle{casual}
\moderncvcolor{blue}
\renewcommand{\familydefault}{\sfdefault}
\nopagenumbers{}

%% Character encoding
\usepackage[utf8]{inputenc}

%% Adjust the page margins
\usepackage[scale=0.85]{geometry}

%% Personal data
\firstname{Julien}
\familyname{Larget-Piet}
\title{github.com/julienlargetpiet} 
\address{Poissy}{78300} 
\email{j.larget-piet@proton.me} 
\homepage{https://www.linkedin.com/in/julien-larget-piet/} 
\photo[64pt][0.4pt]{cv1.jpg}
\quote{\huge Autonomie - Rigueur - Travail en équipe}
%%------------------------------------------------------------------------------
%% Content
%%------------------------------------------------------------------------------
\begin{document}
\makecvtitle

\vspace{-1cm}
\section{Expériences professionnelles et études}

\cventry{2022--2023}{Licence Data Mining}{Université Gustave Eiffel}{Noisy-Champs}{Stage de 6 mois chez Dreamrs}{Développement d'applications sous \textbf{Rshiny}, dashboard sur l'évolution de la consommation électrique, les scores aux J.O...}  
\cventry{2021--2023}{BTS - Métiers de la Chimie}{Lycée Notre Dame Les Oiseaux}{Verneuil-Sur-Seine}{Stage de 2 mois chez Quad Service pour aider à une étude sur les microplastiques}{Développement de logiciels sous \textbf{Python} et \textbf{Bash} et familiarisation avec Linux en parallèle}
\cventry{2021--2022}{Baccalauréat options mathématiques et physique-chimie}{Charles de Gaulle}{Poissy}{}{}  

\section{Connaissances techniques}
\cvlistdoubleitem{Python (Flask, Django, pandas, polars, numpy, scipy...)}{Bash (AWK, Sed, Ps, Grep)}
\cvlistdoubleitem{R (data.table, dplyr, rshiny)}{C++}
\cvlistdoubleitem{HTML}{CSS}
\cvlistdoubleitem{LaTeX (Tikz...)}{Ffmpeg}
\cvlistdoubleitem{Git}{NGINX}


\section{Développement et contributions à l'OpenSource}

\cventry{2023-aujourd'hui}{R et C++}{}{Librairie C++ pour calculer de très grands nombres et une librairie de manipulation de données. Outil de documentation. \textbf{Framework} web pour site personnel avec \textbf{Flask}. Algorithme d'extraction de voix avec python et \textbf{Ffmpeg}. Librairie \textbf{R} pour la manipulation de coordonnées géographiques}{}{}

\cventry{2022-2023}{Python et Bash}{}{Logiciel de comptabilité avec \textbf{openpyxl}. Utilitaires en Bash. Réimplémentation de \textbf{crontab}.}{}{}

\cventry{2021-2022}{Familiarisation avec python}{}{\textbf{Parser} de parenthèses. Apprentissage de \textbf{matplotlib} et les librairies de scrapping comme \textbf{bs4}. J'apprends à rédiger de la documentation. Découverte de l'environnement \textbf{Linux}.}{}{}

\section{Communication et vulgarisation}

\cventry{2022 - aujourd'hui}{Transmission}{}{Je poste sur \textbf{LinkedIn} et \textbf{Youtube} en vulgarisant mon travail et en transmettant ma passion. Je suis aussi sur \textbf{stackoverflow} pour aider ou poser des questions.}{}{}

\section{Langues}
\cvitemwithcomment{Anglais}{Toeic obtenu}{}


\end{document}




























